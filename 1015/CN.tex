\def\draft{0}
\documentclass[11pt]{article}
\usepackage{amsfonts, fullpage, rotating, amssymb}
\usepackage{color,amsmath}
\usepackage{IEEEtrantools}
\usepackage{amsfonts, fullpage, rotating, amssymb}
\usepackage{color,amsmath}
\usepackage{IEEEtrantools}
\usepackage{multirow}
\usepackage{tcolorbox}
\usepackage{bbm}
\usepackage{hyperref}
\newtheorem{theorem}{Theorem}
\newtheorem{conjecture}{Conjecture}
\newtheorem{definition}{Definition}
\newtheorem{lemma}{Lemma}
\newtheorem{proposition}{Proposition}
\newtheorem{property}{Property}
\newtheorem{corollary}{Corollary}
\newtheorem{claim}{Claim}
\newtheorem{fact}{Fact}
\newtheorem{openprob}[theorem]{Open Problem}
\newtheorem{remk}{Remark}
\newtheorem{exmp}{Example}
\newtheorem{apdxlemma}{Lemma}
\newenvironment{example}{\begin{exmp}
\begin{normalfont}}{\end{normalfont}
\end{exmp}}

\newenvironment{remark}{\begin{remk}
\begin{normalfont}}{\end{normalfont}
\end{remk}}
\newtheorem{sublemma}[theorem]{Sublemma}

\newenvironment{intuition}[1][]{\begin{tcolorbox}[title=Intuition ~(#1),fonttitle=\sffamily\large\bfseries]}{\end{tcolorbox}}

%%%%%%%%%%%%%%%%%%%% proof environments

\def\FullBox{\hbox{\vrule width 8pt height 8pt depth 0pt}}

\def\qed{\ifmmode\qquad\FullBox\else{\unskip\nobreak\hfil
\penalty50\hskip1em\null\nobreak\hfil\FullBox
\parfillskip=0pt\finalhyphendemerits=0\endgraf}\fi}

\def\qedsketch{\ifmmode\Box\else{\unskip\nobreak\hfil
\penalty50\hskip1em\null\nobreak\hfil$\Box$
\parfillskip=0pt\finalhyphendemerits=0\endgraf}\fi}

\newenvironment{proof}{\begin{trivlist} \item {\bf Proof:~~}}
  {\qed\end{trivlist}}
\newenvironment{solve}{\begin{trivlist} \item {\bf Solve:~~}}
  {\qed\end{trivlist}}
\newenvironment{proofsketch}{\begin{trivlist} \item {\bf
Proof Sketch:~~}}
  {\qedsketch\end{trivlist}}

\newenvironment{proofof}[1]{\begin{trivlist} \item {\bf Proof
#1:~~}}
  {\qed\end{trivlist}}

\newenvironment{claimproof}{\begin{quotation} \noindent
{\bf Proof of claim:~~}}{\qedsketch\end{quotation}}


%%%%%%%%%%%%%%%%%%%%%%% text macros
\newcommand{\etal}{{\it et~al.\ }}
\newcommand{\ie} {{\it i.e.,\ }}
\newcommand{\eg} {{\it e.g.,\ }}
\newcommand{\cf}{{\it cf.,\ }}

%%%%%%%%%%%%%%%%%%%%%%% general useful macros
\newcommand{\eqdef}{\mathbin{\stackrel{\rm def}{=}}}
\newcommand{\R}{{\mathbb R}} % real numbers
\newcommand{\N}{{\mathbb{N}}} % natural numbers
\newcommand{\Z}{{\mathbb Z}} % integers
\newcommand{\F}{{\mathbb F}} % a field
\newcommand{\Q}{{\mathbb Q}} % the rationals
\newcommand{\poly}{{\mathrm{poly}}}
\newcommand{\polylog}{{\mathrm{polylog}}}
\newcommand{\loglog}{{\mathop{\mathrm{loglog}}}}
\newcommand{\zo}{\{0,1\}}
\newcommand{\suchthat}{{\;\; : \;\;}}
\newcommand{\pr}[1]{\Pr\left[#1\right]}
\newcommand{\deffont}{\em}
\newcommand{\getsr}{\mathbin{\stackrel{\mbox{\tiny R}}{\gets}}}
\newcommand{\Exp}{\mathop{\mathrm E}\displaylimits} % expectation
\newcommand{\Var}{\mathop{\mathrm Var}\displaylimits} % variance
\newcommand{\xor}{\oplus}
\newcommand{\GF}{\mathrm{GF}}
\newcommand{\eps}{\varepsilon}
\newcommand{\limT}{\lim_{T\rightarrow\infty}}
\newcommand{\limn}{\lim_{n\rightarrow\infty}}
\DeclarePairedDelimiter\ceil{\lceil}{\rceil}
\DeclarePairedDelimiter\floor{\lfloor}{\rfloor}
%%%%%%%%%%%%%%%%%%% macros particular to this course


\pagestyle{plain}

\newcommand{\scribe}{Chi-Ning Chou}
\newcommand{\lecnum}{10}
\newcommand{\lecdate}{October 15, 2015}

%\parskip=1.5mm
%\parindent=0mm

\begin{document}
	
	\begin{center}
		\renewcommand{\arraystretch}{2}
		\begin{bfseries}
			\begin{tabular}{|c|}
				\hline
				Statistical Inference I \hfill Prof. Chin-Tsang Chiang\\
				\hspace{15em} {\large Lecture Notes \lecnum} \hspace{15em}\ \\
				\lecdate \hfill Scribe: \scribe\\
				\hline
			\end{tabular}
			\renewcommand{\arraystretch}{1}
		\end{bfseries}
	\end{center}
	
\section{Expectation}
Expectation is simply a functional of the distribution. It maps a distribution to a certain real value to represent the behavior, shape, or other properties. Formally, we define the expectation of a random variable $X$ as follow:
\begin{definition}[expectation]
	Let $X$ be a r.v. and $g$ be a measurable function. Then, the expectation of $g(X)$, which is also a r.v., is denoted as $\mathbb{E}[g(X)]$,\ie
	$$\mathbb{E}[g(X)] = \int_x g(x)dF_X(x)$$
	Note that the expectation of $\mathbb{E}[g(X)]$ exists provided that $\mathbb{E}[|g(X)|]<\infty$.
\end{definition}
{\bf Remark}:
If the distribution is not a mixture of both discrete and continuous distribution, then we can represent it as
\begin{itemize}
	\item If $X$ is discrete, $\mathbb{E}[g(X)] = \sum_x g(x)f_X(x)$.
	\item If $X$ is continuous, $\mathbb{E}[g(X)] = \int_{-\infty}^{\infty}g(x)dF_X(x)$.
\end{itemize}

However, not all the distribution has expectation! Cauchy distribution is a beautiful example:
{\bf Example: }({\bf Cauchy distribution has no mean})\\
The pdf of Cauchy distribution is
$$f(x) = \frac{1}{\pi(1+x^2)}dx$$
With simple integration, we can check that $\mathbb{E}(|X|) = 2\int_0^{\infty}\frac{x}{\pi(1+x^2)}dx=\infty$. Thus, the expectation of Cauchy distribution does not exist. As a remark, Cauchy is a bell-shaped distribution with median 0. And actually, the cumulative distribution of Cauchy is the arc tangent function!

\begin{property}
	Let $X$ be a r.v. and a,b,c be constants. Moreover, $g_1(X)$, $g_2(X)$ be any r.v. with expectation. Then,
	\begin{enumerate}
		\item (Preserve linear combination) $\mathbb{E}[]ag_1(X)+bg_2(X)+c]=a\mathbb{E}[g_1(X)] + b\mathbb{E}[g_2(X)] + c$.
		\item (Preserve non-negativity) If $f(x)\geq0,\ \forall x$, then $\mathbb{E}[g(X)]$.
		\item (Preserve dominance) If $g_1(x)\geq g_2(x),\ \forall x$, then $\mathbb{E}[g_1(X)]gleq\mathbb{E}[g_2(X)].$
		\item (Existence of bounded r.v.) If $a\leq g(x) \leq b,\ \forall x$, then $a\leq \mathbb{E}[g(X)]\leq\mathbb{E}[g(X)] b.$
	\end{enumerate}
\end{property}

Now, we turn to an useful and interesting application of expectation.\\

\noindent{\bf Example}: ({\bf The expectation of indicator function is probability})
Consider $I_A$ to be an indicator function of a set $A\subseteq\R$, then
$$\mathbb{E}[I_A(X)] = P(A)$$
Moreover, we can regard the above equation as a {\bf binary response}. That is, the indicator separate the space $\R$ into two parts: $\{x:x\in A\}$ and $\{x:x\notin A \}$ and the expectation is a functional to see the response of such partition.

For example, consider the following indicator function $I(X\leq x)$. We can see that $\mathbb{E}[I(X\leq x)] = F_X(x)$. And this representation gives us a broad way to describe the data. Suppose now we are concerning the probability $Pr[X=x|Z_1,Z_2,...,Z_p]$, the most simply way is to use a general model to describe it, say
$$Pr[X=x|Z_1,Z_2,...,Z_p] = G(x,\beta_1Z_1+\beta_2Z_2+...+\beta_pZ_p)$$
As we choose to use the expectation representation: $\mathbb{E}[I(X\leq x)|Z_1,Z_2,...,Z_p]$, the impact of $Z_i$s can somehow depends on the value of $x$ and become even more general. In other words, the linear parameter $\beta_i$s can be depended on $x$. For example,
\begin{align*}
x_1: \{x:X\leq x_1 \}\leftrightarrow \beta_{11}Z_1+\beta_{12}Z_2+...+\beta_{1p}Z_p\\
x_2: \{x:X\leq x_2 \}\leftrightarrow \beta_{21}Z_1+\beta_{22}Z_2+...+\beta_{2p}Z_p
\end{align*}

With the above concept, we can simply show the inclusion-exclusion theorem with the help of indicator function and its expectation. First consider two facts:
\begin{itemize}
	\item $\mathbf{1}_{A\cap B} = \mathbf{1}_A\mathbf{1}_B$ and $\mathbf{1}_{A\cup B} = 1-\mathbf{1}_{A^C\cap B^C}$
	\item $\mathbf{1}_{\cup_iA_i} = 1-\prod_i(1-\mathbf{1}_{A_i})$
\end{itemize}
Now, we can derive the inclusion-exclusion theorem:
\begin{align*}
P(\cup A_i) &= 1-\mathbb{E}[\prod_i(1-\mathbf{1}_{A_i})]\\
&= 1 - \mathbb{E}[1-\sum_i\mathbf{1}_{A_i} + \sum_{i,j}\mathbf{1}_{A_i}\mathbf{1}_{A_j}-...+(-1)^k\sum_{i_1,...,i_k}\mathbf{1}_{A_{i_1}}\cdots\mathbf{1}_{A_{i_k}}\pm...\pm\mathbf{1}_{\cap A_i}]\\
&=\sum_i P(A_i) - \sum_{i,j}P(A_i\cap A_j) +...+(-1)^{k-1}\sum_{i_1,...,i_k}P(A_{i_1}\cap\cdots\cap A_{i_k})\pm...\pm P(\cup_i A_i)
\end{align*}


\end{document}
