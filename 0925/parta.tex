\def\draft{0}
\documentclass[11pt]{article}
\usepackage{amsfonts, fullpage, rotating, amssymb}
\usepackage{color,amsmath}
\usepackage{IEEEtrantools}
\usepackage{tcolorbox}
\usepackage{extarrows}
\newtheorem{theorem}{Theorem}
\newtheorem{conjecture}{Conjecture}
\newtheorem{definition}{Definition}
\newtheorem{lemma}{Lemma}
\newtheorem{proposition}[theorem]{Proposition}
\newtheorem{property}{Property}
\newtheorem{corollary}[theorem]{Corollary}
\newtheorem{claim}[theorem]{Claim}
\newtheorem{fact}[theorem]{Fact}
\newtheorem{openprob}[theorem]{Open Problem}
\newtheorem{remk}{Remark}
\newtheorem{exmp}{Example}
\newtheorem{apdxlemma}{Lemma}
\newenvironment{example}{\begin{exmp}
\begin{normalfont}}{\end{normalfont}
\end{exmp}}

\newenvironment{remark}{\begin{remk}
\begin{normalfont}}{\end{normalfont}
\end{remk}}
\newtheorem{sublemma}[theorem]{Sublemma}

\newenvironment{intuition}[1][]{\begin{tcolorbox}[title=Intuition ~(#1),fonttitle=\sffamily\large\bfseries]}{\end{tcolorbox}}


%%%%%%%%%%%%%%%%%%%% proof environments

\def\FullBox{\hbox{\vrule width 8pt height 8pt depth 0pt}}

\def\qed{\ifmmode\qquad\FullBox\else{\unskip\nobreak\hfil
\penalty50\hskip1em\null\nobreak\hfil\FullBox
\parfillskip=0pt\finalhyphendemerits=0\endgraf}\fi}

\def\qedsketch{\ifmmode\Box\else{\unskip\nobreak\hfil
\penalty50\hskip1em\null\nobreak\hfil$\Box$
\parfillskip=0pt\finalhyphendemerits=0\endgraf}\fi}

\newenvironment{proof}{\begin{trivlist} \item {\bf Proof:~~}}
  {\qed\end{trivlist}}
\newenvironment{solve}{\begin{trivlist} \item {\bf Solve:~~}}
  {\qed\end{trivlist}}
\newenvironment{proofsketch}{\begin{trivlist} \item {\bf
Proof Sketch:~~}}
  {\qedsketch\end{trivlist}}

\newenvironment{proofof}[1]{\begin{trivlist} \item {\bf Proof
#1:~~}}
  {\qed\end{trivlist}}

\newenvironment{claimproof}{\begin{quotation} \noindent
{\bf Proof of claim:~~}}{\qedsketch\end{quotation}}


%%%%%%%%%%%%%%%%%%%%%%% text macros
\newcommand{\etal}{{\it et~al.\ }}
\newcommand{\ie} {{\it i.e.,\ }}
\newcommand{\eg} {{\it e.g.,\ }}
\newcommand{\cf}{{\it cf.,\ }}

%%%%%%%%%%%%%%%%%%%%%%% general useful macros
\newcommand{\eqdef}{\mathbin{\stackrel{\rm def}{=}}}
\newcommand{\R}{{\mathbb R}} % real numbers
\newcommand{\N}{{\mathbb{N}}} % natural numbers
\newcommand{\Z}{{\mathbb Z}} % integers
\newcommand{\F}{{\mathbb F}} % a field
\newcommand{\Q}{{\mathbb Q}} % the rationals
\newcommand{\poly}{{\mathrm{poly}}}
\newcommand{\polylog}{{\mathrm{polylog}}}
\newcommand{\loglog}{{\mathop{\mathrm{loglog}}}}
\newcommand{\zo}{\{0,1\}}
\newcommand{\suchthat}{{\;\; : \;\;}}
\newcommand{\pr}[1]{\Pr\left[#1\right]}
\newcommand{\deffont}{\em}
\newcommand{\getsr}{\mathbin{\stackrel{\mbox{\tiny R}}{\gets}}}
\newcommand{\Exp}{\mathop{\mathrm E}\displaylimits} % expectation
\newcommand{\Var}{\mathop{\mathrm Var}\displaylimits} % variance
\newcommand{\xor}{\oplus}
\newcommand{\GF}{\mathrm{GF}}
\newcommand{\eps}{\varepsilon}

%%%%%%%%%%%%%%%%%%% macros particular to this course


\pagestyle{plain}
\newcommand{\scribe}{Wei-Chang Lee}
\newcommand{\lecnum}{4}
\newcommand{\lecdate}{September 27, 2015}

%\parskip=1.5mm
%\parindent=0mm

\begin{document}
	
	\begin{center}
		\renewcommand{\arraystretch}{2}
		\begin{bfseries}
			\begin{tabular}{|c|}
				\hline
				Statistical Inference I \hfill Prof. Chin-Tsang Chiang\\
				\hspace{15em} {\large Lecture Notes \lecnum} \hspace{15em}\ \\
				\lecdate \hfill Scribe: \scribe\\
				\hline
			\end{tabular}
			\renewcommand{\arraystretch}{1}
		\end{bfseries}
	\end{center}
	
Today we're going to talk about method of counting to be able to study probability assignments on finite space, we will then give definition about conditional probability on discrete case and intuition about continuous case. After knowing how to calculate conditional probabilities, we can then use Bayes' theorem to connect experimental study to observation study which is called the odds ratio. At the end of the class, Prof.Chiang introduce the concept "independence" and how to interpret it beyond geometric views in a more statistical way.
\section{Counting Methods}
  There are two kinds of interpretation of probability measure. 
\begin{enumerate}
\item[1]
  The first kind of view is based on the "frequency of occurrence". They do random experiments many times to study  how many times event which is interested in would take place. Then, they assign the limit ratio  as the probability of certain events. The intuition of this interpretation is that we believe the underlying parameter is invariant so after many repeated experiment the random effect can be cancelled out and the real instinct reveal. 
\item[2]
  The second kind of view is just based on the "subjective belief of interpreter" in the chance of an event occurring.  We can just give probability under our faith before any experiment has been done and then do experiment to study whether our prior assumption fit the data or not and then modify it. This procedure can be done by Bayes' theorem. 
\end{enumerate}
\paragraph{}
  The counting problem is sometimes sophisticated and along with many restrictions. The way to solve such problem is to break them into series of simple tasks and apply rules to combine it back. 
\begin{theorem}[Fundamental Theorem of Counting]
	Suppose a job consists of k separate tasks, the $i$th of which can be done in $n_i$ ways, then entire job can be done in $\prod_{i=1}^k n_i$ ways.
\end{theorem}
\begin{intuition}[Classification of counting methods]
	The proof of Fundamental Theorem of Counting is quite trivial. Sometimes it is better to think of task as partition criteria such as love or hate, different gender or income level then we just construct a sample space $\Omega=\Omega_1\times\Omega_2...\Omega_k$ which is the cartesian product of k criteria we interested in then we can apply the theorem to calculate there are how many possible outcomes.\\
In reality, we may face situation such as replacement and unordered. Replacement means that $n_i=n_{i+1}$ and unordered means that we can perform k separate tasks arbitrary without certain order. In such case we have to carefully apply or modify the fundamental counting theorem. 	
\end{intuition}
Consider number of possible arrangements of size r drawing from n different subjects. We can divide the case into four categories.
\begin{enumerate}
	\item Without replacement and ordered: $n*(n-1)*(n-2)...(n-r+1)=\prod_{i=1}^r(n-i+1)=\frac{n!}{(n-r)!}$
	\item With replacement and ordered:$n*n*n...n=\prod_{i=1}^r n=n^r$
	\item Without replacement and unordered: $\frac{\prod_{i=1}^r(n-i+1)}{r!}=\frac{n!}{(n-r)!(r!)}=\binom{n}{r}$ 
	\item With replacement and unordered: $\binom{n+r-1}{r}$
\end{enumerate}
\begin{proof}
Case 1 and case 2 is quite simple just applying the Fundamental Theorem of Counting. For case 3, r different objects can be permuted in $r*(r-1)*(r-2)...1=r!$ ways but they represent the same arrangement in unordered situation. So we have case 3 equals case 1 divided by r!. Case 4 is the most difficult, we may simply view it as case 2 divided by r! but it will underestimate the possible arrangements since r objects with some of them are of same kind do not construct r! different permutations. A clever way to solve case 4 is to think of r as numbers of coins and place it arbitrary but all into n different box. A coin in $i$th box means in our ultimate arrangements we have one $i$th objects. Consider a small case as n=3 and r=3. Then the following figure is the realization of picking two $2$th object and one $3$th object. And all possible realization cab be expressed as all possible arrangements of 2 $|$ and 3 $O$ which is $\binom{3-1+3}{3}$. 
$$\underline{\quad}^1 |\underline{OO}^2 |\underline{O}^3$$ So for case 4, the answer is $\binom{n-1+r}{r}$.
\end{proof}
\paragraph{}
The counting techniques are useful when the sample space is finite and every possible outcomes in S are equally likely. The probability of certain event can be calculated by the number of outcomes in that event times the probability of each outcome from the countably addictive axiom.
\begin{theorem}[Enumerating outcomes]
Let $\Omega=\{w1,w2...wn\}$ with $P(\{w_i\})=\frac{1}{n}$ $\forall i$ then $P(A)=\sum_{\{w_i\}\in A}P(\{w_i\})=\sum_{\{w_i\}\in A}\frac{1}{n}=\frac{\sharp(A)}{\sharp(\Omega)}$ $\forall A$.
\end{theorem}
\begin{remark}
This is also the classical definition of probability from Pierre-Simon Laplace.
\end{remark} 
\section{Conditional probability}

\end{document}




