\documentclass[../Distributions.tex]{subfiles}
\begin{document}
	\begin{center}
		\renewcommand{\arraystretch}{2}
		\begin{bfseries}
			\begin{tabular}{|c|}
				\hline
				Statistical Inference I \hfill Prof. Chin-Tsang Chiang\\
				\hspace{15em} {\large Lecture Notes 11} \hspace{15em}\ \\
				\lecdate \hfill Scribe: \scribe\\
				\hline
			\end{tabular}
			\renewcommand{\arraystretch}{1}
		\end{bfseries}
	\end{center}
	
\section{Discrete Distributions}

\begin{definition}[Discrete Uniform Distribution]Suppose X follows discrete uniform distribution then it has density $$f_X(|N)=\frac{1}{N}\mathbbm{1}_{\{1,2,3...N\}}(x)$$
where N is an integer, with notation X$\sim$Discrete Uniform(N).
\end{definition}

\begin{property}Given N, X follows discrete uniform distribution then,
\begin{enumerate}
\item $E[X|N]=\sum_{i=1}^NP(X=i)i=\frac{N+1}{2}$
\item $Var(X|N)=E[X^2|N]-E[X|N]^2=\frac{N^2-1}{12}$
\end{enumerate}
\end{property}
\begin{intuition}[Usage in statistics]
How can we test the two given data group X,Y follows the same distribution? \\
$X_1,X_2...X_n \overset{iid}{\sim}F_1(x)$ and $Y_1,Y_2...Y_m  \overset{iid}{\sim}F_1(x)$ want to test $H_0:F_1(x)=F_2(x)\ \forall x$\\
{\bf Kolmogrov statistics}:
Using empirical distribution of $F_1(x),F_2(x)$
$$\hat{F_1}(x)=\frac{1}{n}\sum_{i=1}^n\mathbbm{1}(X_i\leq x)$$
$$\hat{F_2}(x)=\frac{1}{m}\sum_{i=1}^m\mathbbm{1}(Y_i\leq x)$$
We have Kolmogrov statistics:$$\sup_x|\hat{F_1}(x)-\hat{F_2}(x)|$$ which can not be too big if X and Y following same distributions.\\
{\bf Rank statistics(Wilcoxon test)}: Instead of using true value as the predictor. We use the order i.e. rank of the data in the group. We combine X and Y and sort them to give rank $$W=\frac{1}{n}\sum_{i=1}^{n}Rank(X_i)$$
To prevent the issue that extreme values dominated the statistics. And $X\sim Y$ if W is not too big or too small. 

\end{intuition}
\begin{definition}[Bernoulli Distribution]X follows Bernoulli distribution then it has density
$$f_X(x|p)=p^x(1-p)^{1-x}\mathbbm{1}_{\{0,1\}}(x)$$
where $0\leq p\leq 1$ denoting as X$\sim$Bernoulli(p).
\end{definition}
\begin{property}Given p, X follows binomial distribution then,
\begin{enumerate}
\item $E[X^m|p]=E[X|p]=p$
\item $Var(X|p)=p-p^2=p(1-p)$
\item $F_X(x)=P(X\leq x)=E[I(X\leq x)]=E[N(x)]$
\end{enumerate}
\end{property}
\begin{definition}[Binomial Distribution]$X_1,X_2...X_n$ i.i.d follows Bernoulli(p), let $X=\sum_{i=1}^nX_n$, X follows binomial distribution having density $$f_X(x)=\binom{n}{x}p^x(1-p)^x\mathbbm{1}_{\{0,1,2,3...n\}}(x)$$
\end{definition}
\begin{intuition}[Independent]
$X_1,X_2...X_n$ are independent iff $$P(X_1=x_1,X_2=x_2...X_n=x_n)=\prod_{i=1}^nf(x_i|p)$$
The mutually independent property automatically satisfied since we can think of $\Omega=\Omega_1\times\Omega_2...\times\Omega_n$ where $\Omega_i=\{0,1\}$ for the $i$th Bernoulli trial. And $\bigcup(A_i\in\Omega_1)(\Omega_2...\times\Omega_n)$ augmented $\Omega$.
\end{intuition}
\begin{remark}In reality, $X_1,X_2...X_n$ are not i.i.d.. Since we sometimes sample population with common factors. They may affect each other, within positive or negative relation.
\begin{enumerate}
\item Over-dispersion binomial distribution: There are positive correlation among populations. That is, if the event happens on one member, then other members will have higher tendency to success.
\item Under-dispersion binomial distribution: There are negative correlation among populations.
\end{enumerate}
Formally, if the variance of a random variable look like: $var[X]=\phi p(1-p)$. If $\phi > 1$, we say $X$ is a over-dispersion binomial and on the contrary if $\phi<1$, then we say $X$ is an under-dispersion binomial. Note that, although we call them "binomial", they are definitely not a binomial random variable!

\end{remark}

\end{document}
