\documentclass[../Distributions.tex]{subfiles}
\begin{document}
	\begin{center}
		\renewcommand{\arraystretch}{2}
		\begin{bfseries}
			\begin{tabular}{|c|}
				\hline
				Statistical Inference I \hfill Prof. Chin-Tsang Chiang\\
				\hspace{15em} {\large Lecture Notes 15} \hspace{15em}\ \\
				\lecdate \hfill Scribe: \scribe\\
				\hline
			\end{tabular}
			\renewcommand{\arraystretch}{1}
		\end{bfseries}
	\end{center}

\section{Logistic distribution}
The cumulative distribution function of logistic distribution is
$$F_X(x) = \frac{e^x}{1+e^x}\mathbf{1}_{(-\infty,\infty)}(x)$$
The importance of logistic distribution is that it lies between Cauchy distribution and Gaussian distribution. That is, the decaying rate of logistic distribution is somewhere between $O(2^{-n})$ and $O(2^{-n^2})$.\\

Another important application of logistic distribution is {\it logistic regression}. Here, we sketch the formulation of logistic regression:
\begin{itemize}
	\item Binary response: $Y\in\{0,1\}$
	\item Explanatory variables: $Z_1,...,Z_p$
	\item Odds ratio: $\frac{P(Y=1|Z_1,...,Z_p)}{1-P(Y=1|Z_1,...,Z_p)}$
	\item Logistic transformation:
	$$\ln \frac{P(Y=1|Z_1,...,Z_p)}{1-P(Y=1|Z_1,...,Z_p)} = \beta_0 + \beta_1Z_1 + ... + \beta_p Z_p$$
	\item Positive probability:
	$$P(Y=1|Z_1,...,Z_p) = \frac{e^{\beta_0 + \beta_1Z_1 + ... + \beta_p Z_p}}{1+e^{\beta_0 + \beta_1Z_1 + ... + \beta_p Z_p}}$$
\end{itemize}
To sum up, logistic regression is a special case of generalized linear model and aims to predict the probability of certain outcome.
{\bf Generalized linear model}:
$$P(Y|Z_1,..,Z_p) = F_0(\beta_0+\beta_1Z_1+...+\beta_p Z_p)$$
, where $F_0$ is a cumulative distribution function.

\section{Beta distribution}
Beta distribution is a distribution that describes the battle between 0 and 1. We can create various of distribution in the interval [0,1] with beta distribution. With this abundance property, beta distribution can be used as a prior function in Bayesian analysis. Formally speaking, the density function of beta distribution is
$$f_X(x|\alpha,\beta) = \frac{x^{\alpha-1}(1-x)^{\beta-1}}{\mbox{Beta}(\alpha,\beta)}\mathbf{1}_{[0,1]}(x)$$
, where $\alpha,\beta>0$ and $\mbox{Beta}(\alpha,\beta) = \frac{\Gamma(\alpha)+\Gamma(\beta)}{\Gamma(\alpha+\beta)}$.\\
Now, let's see some basic properties of beta distribution:
\begin{itemize}
	\item $\mathbb{E}[X|\alpha,\beta] = \frac{\alpha}{\alpha+\beta}$
	\item $Var[X|\alpha,\beta] = \frac{\alpha\beta}{(\alpha+\beta)^2(\alpha\beta+1)}$
	\item Shape:
	\begin{itemize}
		\item $\alpha>1,\beta=1$, increasing.
		\item $\alpha=1, \beta > 1$, decreasing.
		\item $\alpha<1,\beta<1$, U shape.
		\item $\alpha>1\beta>1$, unimodal.
		\item $\alpha=\beta$, symmetric.
		\item $\alpha=\beta=1$, uniform.
	\end{itemize}
	\item The relation between beta and binomial: $X\sim$Beta($\alpha,beta$), $Y\sim$Binomial($n,p$)
	$$P(X\geq p|x+1,n-x) = P(Y\leq x|n,[])$$
\end{itemize}

\section{Double exponential distribution (Laplace distribution)}
The density function of Laplace distribution is
$$f_X(x|\mu,\sigma^2) = \frac{1}{2\sigma}e^{-|x-\mu|/\sigma}\mathbf{1}_{(-\infty,\infty)}$$
The following is some basic properties:
\begin{itemize}
	\item $\mathbb{E}[X|\mu,\sigma^2] = \mu$
	\item $Var[X|\mu,\sigma^2] = 2\sigma^2$
\end{itemize}
Note that here $\sigma$ is not variance.

\section{Log-normal distribution}
Let $X$ be a log-normal distribution with parameter $(\mu,\sigma^2)$, then
$$Y = \ln X \sim N(\mu,\sigma^2)$$
The following is some basic properties:
\begin{itemize}
	\item $\mathbb{E}[X|\mu,\sigma^2] \neq\mu$
	\item median = $\mu$
	\item pdf: $f_X(x|\mu,\sigma^2) = (2\pi\sigma^2)^{-1/2}\frac{1}{x}e^{-(\ln x-\mu)^2/2\sigma^2}\mathbf{1}_{(0,\infty)}(x)$
	\item Have moments but no mgf.
	\item $\mathbb{E}[X|\mu,\sigma^2] = e^{\mu+\sigma^2/2}$
	\item $Var[X|\mu,\sigma^2] = (e^{\sigma^2}-1)e^{2\mu+\sigma^2}$
\end{itemize}

\section{Cauchy distribution}
The density function of Cauchy distribution is
$$f_X(x|\mu,\sigma) = \frac{1}{\pi\sigma(1+(\frac{x-\mu}{\sigma})^2)}\mathbf{1}_{(-\infty,\infty)}(x)$$
The following is some basic properties
\begin{itemize}
	\item $\mathbb{E}[|X|\ |\ \mu,\sigma^2] = \infty$
	\item median = $\mu$
	\item If $X,Y$ are two independent $N(0,1)$, then $\frac{X}{Y}\sim$Cauchy(0,1).
\end{itemize}



\end{document}
