\def\draft{0}
\documentclass[11pt]{article}
\usepackage{amsfonts, fullpage, rotating, amssymb}
\usepackage{color,amsmath}
\usepackage{IEEEtrantools}
\usepackage{amsfonts, fullpage, rotating, amssymb}
\usepackage{color,amsmath}
\usepackage{IEEEtrantools}
\usepackage{multirow}
\usepackage{tcolorbox}
\usepackage{bbm}
\usepackage{hyperref}
\newtheorem{theorem}{Theorem}
\newtheorem{conjecture}{Conjecture}
\newtheorem{definition}{Definition}
\newtheorem{lemma}{Lemma}
\newtheorem{proposition}{Proposition}
\newtheorem{property}{Property}
\newtheorem{corollary}{Corollary}
\newtheorem{claim}{Claim}
\newtheorem{fact}{Fact}
\newtheorem{openprob}[theorem]{Open Problem}
\newtheorem{remk}{Remark}
\newtheorem{exmp}{Example}
\newtheorem{apdxlemma}{Lemma}
\newenvironment{example}{\begin{exmp}
\begin{normalfont}}{\end{normalfont}
\end{exmp}}

\newenvironment{remark}{\begin{remk}
\begin{normalfont}}{\end{normalfont}
\end{remk}}
\newtheorem{sublemma}[theorem]{Sublemma}

\newenvironment{intuition}[1][]{\begin{tcolorbox}[title=Intuition ~(#1),fonttitle=\sffamily\large\bfseries]}{\end{tcolorbox}}

%%%%%%%%%%%%%%%%%%%% proof environments

\def\FullBox{\hbox{\vrule width 8pt height 8pt depth 0pt}}

\def\qed{\ifmmode\qquad\FullBox\else{\unskip\nobreak\hfil
\penalty50\hskip1em\null\nobreak\hfil\FullBox
\parfillskip=0pt\finalhyphendemerits=0\endgraf}\fi}

\def\qedsketch{\ifmmode\Box\else{\unskip\nobreak\hfil
\penalty50\hskip1em\null\nobreak\hfil$\Box$
\parfillskip=0pt\finalhyphendemerits=0\endgraf}\fi}

\newenvironment{proof}{\begin{trivlist} \item {\bf Proof:~~}}
  {\qed\end{trivlist}}
\newenvironment{solve}{\begin{trivlist} \item {\bf Solve:~~}}
  {\qed\end{trivlist}}
\newenvironment{proofsketch}{\begin{trivlist} \item {\bf
Proof Sketch:~~}}
  {\qedsketch\end{trivlist}}

\newenvironment{proofof}[1]{\begin{trivlist} \item {\bf Proof
#1:~~}}
  {\qed\end{trivlist}}

\newenvironment{claimproof}{\begin{quotation} \noindent
{\bf Proof of claim:~~}}{\qedsketch\end{quotation}}


%%%%%%%%%%%%%%%%%%%%%%% text macros
\newcommand{\etal}{{\it et~al.\ }}
\newcommand{\ie} {{\it i.e.,\ }}
\newcommand{\eg} {{\it e.g.,\ }}
\newcommand{\cf}{{\it cf.,\ }}

%%%%%%%%%%%%%%%%%%%%%%% general useful macros
\newcommand{\eqdef}{\mathbin{\stackrel{\rm def}{=}}}
\newcommand{\R}{{\mathbb R}} % real numbers
\newcommand{\N}{{\mathbb{N}}} % natural numbers
\newcommand{\Z}{{\mathbb Z}} % integers
\newcommand{\F}{{\mathbb F}} % a field
\newcommand{\Q}{{\mathbb Q}} % the rationals
\newcommand{\poly}{{\mathrm{poly}}}
\newcommand{\polylog}{{\mathrm{polylog}}}
\newcommand{\loglog}{{\mathop{\mathrm{loglog}}}}
\newcommand{\zo}{\{0,1\}}
\newcommand{\suchthat}{{\;\; : \;\;}}
\newcommand{\pr}[1]{\Pr\left[#1\right]}
\newcommand{\deffont}{\em}
\newcommand{\getsr}{\mathbin{\stackrel{\mbox{\tiny R}}{\gets}}}
\newcommand{\Exp}{\mathop{\mathrm E}\displaylimits} % expectation
\newcommand{\Var}{\mathop{\mathrm Var}\displaylimits} % variance
\newcommand{\xor}{\oplus}
\newcommand{\GF}{\mathrm{GF}}
\newcommand{\eps}{\varepsilon}
\newcommand{\limT}{\lim_{T\rightarrow\infty}}
\newcommand{\limn}{\lim_{n\rightarrow\infty}}
\DeclarePairedDelimiter\ceil{\lceil}{\rceil}
\DeclarePairedDelimiter\floor{\lfloor}{\rfloor}
%%%%%%%%%%%%%%%%%%% macros particular to this course


\pagestyle{plain}

\newcommand{\scribe}{Chi-Ning Chou}
\newcommand{\lecnum}{12}
\newcommand{\lecdate}{October 26, 2015}

%\parskip=1.5mm
%\parindent=0mm

\begin{document}
	
	\begin{center}
		\renewcommand{\arraystretch}{2}
		\begin{bfseries}
			\begin{tabular}{|c|}
				\hline
				Statistical Inference I \hfill Prof. Chin-Tsang Chiang\\
				\hspace{15em} {\large Lecture Notes \lecnum} \hspace{15em}\ \\
				\lecdate \hfill Scribe: \scribe\\
				\hline
			\end{tabular}
			\renewcommand{\arraystretch}{1}
		\end{bfseries}
	\end{center}
	
	
Today we talk about the discrete distributions related to Bernoulli distribution.
\section{Big picture}
Bernoulli distribution is a single event with two possible outcome: yes/no. The probability is $p$ for the yes result and $(1-p)$ for the no. Intuitively, we can view a Bernoulli distribution as an indicator to identify whether an event has happened.

$$\mbox{\it What if we want to consider more than one event?}$$

Imagine the following scenario, there is a large population containing $N$ elements and $M$ of them are label as {\it type-I} and the rest $N-M$ are labeled as {\it type-II}. Now, as a statistician, we want to draw some inference about the population, but we have only limited access to the population, say $k$ samples. What can we know from the experiment?

Basically, we can categorize the above scenario with two different properties: 
\begin{itemize}
	\item Draw {\it with replacement} ot {\it without replacement}.
	\item Draw {\it fix number of samples}, or keep drawing {\it until a certain event happens}?
\end{itemize}

With these two factors, we can extend Bernoulli distribution into the following three discrete distribution:

\begin{table}[h]
	\centering
	\begin{tabular}{|l|c|c|c|}
		\hline
		& \multicolumn{1}{l|}{Replacement} & \multicolumn{1}{l|}{Draw} & \multicolumn{1}{l|}{Goal} \\ \hline
		Hypergeometric    & Without                          & $k$ times                 & Number of yes             \\ \hline
		Binomial          & With                             & $k$ times                 & Number of yes             \\ \hline
		Negative Binomial & With                             & Wait until $r$ yes        & Number of no              \\ \hline
	\end{tabular}
\end{table}

\section{Hypergeometric distribution}
\subsection{Definition}
Hypergeometric distribution describes the probability of the number of {\it yes} result under $k$ samples {\bf without replacement}. The density function consists of three parameters: ($N,M,k$) and the pdf is
$$f(x|N,M,k) = \frac{{M\choose x}{N-M \choose k-x}}{{N\choose k}}\mathbf{1}_{(\max(0,k-(N-M)),min(M,k))}(x)$$
Here, we discuss the meaning of each term:
\begin{itemize}
	\item ${N\choose k}$ in the denominator is the number of possible $k$ samples outcome.
	\item ${M\choose x}$ in the numerator is the number of possible combinations of $k$ yes instances.
	\item ${N-M\choose M-x}$ in the numerator is the number of possible combinations of $x-k$ no instances.
\end{itemize}

\subsection{Basic properties}
Here, we list the mean and variance of hypergeometric distribution and discuss the idea of reparametrize techniques.
\begin{itemize}
	\item $\mathbb{E}[X|N,M,k] = k\frac{M}{N}$
	\item $var[X|N,M,k]=k\frac{M}{N}\frac{N-M}{N}\frac{N-k}{N-1}$
\end{itemize}
In the following, we are going to prove the above results via reparametrize techniques and factorial moment.
{\bf Proof}:
\begin{itemize}
	\item The mean of $X\sim\textit{Hypergeometric}(N,M,k)$
	\begin{align*}
	\mathbb{E}[X|N,M,k] &= \sum_{x=\max(0,k-(N-M))}^{\min(M,k)} x \frac{{M\choose x}{N-M \choose k-x}}{{N\choose k}}= \sum_{x=\max(0,k-(N-M))}^{\min(M,k)} M \frac{{M-1\choose x-1}{N-M \choose k-x}}{{N\choose k}}\\
	&= \sum_{x=\max(0,k-(N-M))}^{\min(M,k)} M \frac{{M-1\choose x-1}{N-(M-1) \choose (k-1)-(x-1)}}{{N-1\choose k-1}\times\frac{N}{k}}\\
	&=k\frac{M}{N}\sum_{x}\frac{{M-1\choose x-1}{N-(M-1) \choose (k-1)-(x-1)}}{{N-1\choose k-1}}=k\frac{M}{N}
	\end{align*}
	\item The variance of $X\sim\textit{Hypergeometric}(N,M,k)$
	\begin{align*}
	var[X|N,M,k] = \mathbb{E}[X^2]-\mathbb{E}[X]^2 = \mathbb{E}[X(X-1)] + \mathbb{E}[X] - \mathbb{E}[X]^2
	\end{align*}
	As we know $\mathbb{E}[X]$, it suffices to find $\mathbb{E}[X(X-1)]$. The trick that computing the expectation of $X(X-1)$ instead of that of $X^2$ is called {\it factorial moment}, which is computation-friendly when having lots of binomial terms. As a result,
	\begin{align*}
	\mathbb{E}[X(X-1)] &= 
	\end{align*}
\end{itemize}

\section{Binomial distribution}
\subsection{Definition}
The binomial distribution describe the probability of the number of {\it yes} results with a fixed number of i.i.d. drawing {\bf with replacement}. The density function consists of two parameters: ($N,p$) and the pdf is
$$f(x|N,p) = {N\choose x}p^x(1-p)^{N-x}\mathbf{1}_{0,1,...,N}(x)$$

Note that, the difference between the definitions of hypergeometric distribution and binomial distribution is not only with/without replacement, the underlying mechanism of binomial distribution is not a fix finite sample space as hypergeometric. For example, the number of drawing can be unbounded, or the {\it yes} probability should not be necessarily a rational number.

\subsection{Basic properties}
Suppose $X\sim\textit{Binomial}(N,p)$, the following is the mean and variance of $X$:
\begin{itemize}
	\item $\mathbb{E}[X|N,p] = Np$
	\item $var[X|N,p] = Np(1-p)$
\end{itemize}

\section{Negative binomial distribution}
\subsection{Definition}
The negative binomial distribution describes the probability of the number of {\it no} instances before certain number of {\it yes} results in a sequence of i.i.d. drawing. Formally speaking, for a negative binomial distribution with parameters: ($p,r$) where $p$ is the probability of {\it yes} and $r$ is the number of {\it yes} instances we are waiting for, the pdf is
$$f(x|p,r) = {x+r-1\choose r-1}p^r(1-p)^x\mathbf{1}_{0,1,...}(x)$$

\subsection{Basic properties}
Suppose $X\sim\textit{Negative Binomial}(p,r)$
\begin{itemize}
	\item $\mathbb{E}[X|p,r] = \frac{pr}{1-p}$
	\item $var[X|p,r]=\frac{pr}{(1-p)^2}$
	\item When $r=1$, it is called {\it geometric} distribution.
	\item The drawing process is memoryless. For example, the distribution of number of {\it no} will remain the same as we conditioned on the number of {\it no} instances before.
	\item As we let $p\rightarrow1$, the {\it yes} result will tend to happen and some how the distribution will converge to Poisson distribution similarly to binomial distribution. (Detail discussion next time)
\end{itemize}

\end{document}
