\documentclass[Probability_Theory.tex]{subfiles}
\begin{document}
	\begin{center}
		\renewcommand{\arraystretch}{2}
		\begin{bfseries}
			\begin{tabular}{|c|}
				\hline
				Statistical Inference I \hfill Prof. Chin-Tsang Chiang\\
				\hspace{15em} {\large Lecture Notes 5} \hspace{15em}\ \\
				\lecdate \hfill Scribe: \scribe\\
				\hline
			\end{tabular}
			\renewcommand{\arraystretch}{1}
		\end{bfseries}
	\end{center}

After explaining the concept of independence, today we can finally introduce the Second Borel-Cantelli Lemma. A lemma that can lately be used to prove one of the most important result in probability theory, the strong law of large numbers. Also, we talk about another fundamental concept in probability, random variable, which let us change our views from set to the real field.   

\section{The Second Borel-Cantelli Lemma}
Also called the reverse Borel-Cantelli Lemma.
\begin{theorem}
Let $A_1,A_2,A_3...A_n...\in \mathcal{A}$ be mutually independent. If $\sum_{i=1}^{\infty}P(A_i)=\infty$, then	$$P(A_{n.i.o}) = 1$$
	, where $A_{n.i.o} = \limsup_{n\rightarrow\infty}A_n.$
\end{theorem}
\begin{proof}
\end{proof}
\section{Random Variables}
\begin{definition}[image of set function]
	Let $X:\Omega\rightarrow\mathcal{X}\subset\R$ be a set function. Then the preimage of $S\subset\R$ over $X$ is
	$$X^{-1}:=\{w\in\Omega:X(w)\in S \}$$
\end{definition}

\begin{property}[Properties of set function]\label{propertyofsetfuncs}
	Let $X:\Omega\rightarrow\mathcal{X}\subset\R$ be a set function, then the following properties hold. (Note that here $X$ is not necessary a random variable)
	\begin{enumerate}
		\item (Close under complementation): $X^{-1}(S^C) = (X^{-1}(S))^C$
		\item (Close under union): $X^{-1}(\bigcup_{\alpha\in\Gamma}S_{\alpha}) = \bigcup_{\alpha\in\Gamma}X^{-1}(S_i)$, where $\{S_{\alpha}\}_{\alpha\in\Gamma}$
		\item (Close under intersection): $X^{-1}(\bigcap_{\alpha\in\Gamma}S_{\alpha}) = \bigcap_{\alpha\in\Gamma}X^{-1}(S_i)$, where $\{S_{\alpha}\}_{\alpha\in\Gamma}$
	\end{enumerate}
\end{property}

By definition, a random variable should satisfy the condition that the preimage of every Borel set should in the event space $\mathcal{A}$. However, it's difficult to check since it's hard to enumerate all Borel set and claim the results. Thus, we would like to find a relaxed but necessary condition for a set function to be a random variable. And the following theorem does so.

\begin{theorem}[necessary and sufficient condition for random variable]\label{rvnscondition}
	Let $X:\Omega\rightarrow\mathcal{X}\subset\R$ be a set function, then
	$$X\mbox{ is a random variable}\Leftrightarrow \forall x\in\R,\ \{w:X(w)\leq x\}\in\mathcal{A}$$
\end{theorem}
The proof is left in Appendix~\ref{sec:proofrvnscondition}

\section{From Set Function to Value Function}
Note that random variable is a function that helps us map the event set $\mathcal{A}$ onto the Borel set. The importance here is that know we can do the equivalent operation on {\bf real number} instead of arbitrary $\sigma$-algebra. This not only provides us a general and uniform way to play but also gives us the opportunity to operate on an {\bf ordered} set.

Soon, we might wonder can we play with an even more general function: value function instead of just a set function onto real number? And by the construction of random variable, there are two direct value functions that play an important role in probability theory.
\begin{theorem}[value functions]
	Let $X$ be a random variable w.r.t. $(\Omega,\mathcal{A},P)$, then we can define
	\begin{itemize}
		\item The measure function of $X$ is $\mu_X:\mathcal{B}\rightarrow\mathbf{R}^+$ such that $$\forall B\in\mathcal{B},\ \mu_X(B):=P\{w:X(w)\in B\}$$
		\item The cumulative distribution of $X$ is $F_X:\mathbf{R}\rightarrow\mathbf{R}^+$ such that $$\forall x\in\mathbf{R},\ F_X(x):=P\{w:X(w)\leq x \}$$
	\end{itemize}
\end{theorem}

Here, we also have a necessary and sufficient condition for cumulative function. Intuitively, when a function $F$ satisfies the following conditions, then there's a random variable with its unique cumulative distribution being $F$.

\begin{theorem}[necessary and sufficient condition of cumulative distribution]
	$F$ is a cumulative distribution iff
	\begin{itemize}
		\item (upper and lower bound) $\lim_{x\rightarrow\infty}F(x)=0,\ \lim_{x\rightarrow\infty}F(x) = 1$
		\item (non-decreasing) $\forall x\leq y$, $F(x)\leq F(y)$
		\item (right continuous) $\lim_{x\rightarrow x_0^+} = F(x_0)$
	\end{itemize}
\end{theorem}
The proof is left in Appendix~\ref{sec:proofcfnscondition}.


\begin{intuition}[set function and value function]
	Careful with the difference of 
	\begin{itemize}
		\item ({\bf random variable}): $X:\Omega\rightarrow S\subset\R$
		\item ({\bf measure function}): $\mu_X:\mathcal{B}\rightarrow\R^+$
		\item ({\bf cumulative distribution}): $F_X:\R\rightarrow\R^+$
	\end{itemize}
\end{intuition}


\appendix
\section{Proof of the necessary and sufficient condition of random variable}\label{sec:proofrvnscondition}
Recall that Theorem~\ref{rvnscondition} provides a necessary and sufficient condition for a set function to be a random variable. Here, we prove the correctness of the theorem.

($\Rightarrow$) First, we can see that $\{w:X(w)\leq x\} = X^{-1}[(-\infty,x)]$. Thus, it's sufficient to show that $(-\infty,x)$ is in Borel set $\forall x$. And this is simple since $(-\infty,x) = \bigcup_{n\in\mathbb{N}}[-n,x]\in\mathcal{B}$.

($\Leftarrow$) This direction can be proved in two steps:
\begin{enumerate}
	\item Show that $\forall a<b,\ X^{-1}([a,b])\in\mathcal{A}$.
	\item Show that $\xi = \{S:\exists w\in\Omega,\ X(w)=S \}$ is a $\sigma$-algebra.
\end{enumerate}
With these two results, we can see that the image of $X$ contains $\mathcal{B}$. Thus, $X$ is a random variable. The following show the correctness of these two:
\begin{enumerate}
	\item $\forall a<b$, consider
	\begin{align*}
	X^{-1}([a,b]) &= X^{-1}((-\infty,b]\backslash(-\infty,a]) = X^{-1}((-\infty,b]\cap(-\infty,a])^C)\\
	&= X^{-1}((-\infty,b]) \cap X^{-1}((-\infty,a])^C)\\
	&=  X^{-1}((-\infty,b]) \cap X^{-1}((-\infty,a]))^C\in\mathcal{A}
	\end{align*}
	\item Check the three axioms of $\sigma$-algebra:
	\begin{enumerate}
		\item Clearly, $\emptyset\in\xi$.
		\item $\forall S\in\xi$, $X^{-1}(X^C) = X^{-1}(S)^C\in\mathcal{A}$ by Property~\ref{propertyofsetfuncs}. Thus $S^C\in\xi$.
		\item $\forall \{S_{\alpha}\}_{\alpha\in\Gamma}\in\xi$, $X^{-1}(\bigcup_{\alpha\in\Gamma}S_{\alpha}) = \bigcup_{\alpha\in\Gamma}X^{-1}(S_{\alpha})\in\mathcal{A}$. Thus $\bigcup_{\alpha\in\Gamma}S_{\alpha}\in\xi$.
	\end{enumerate}
\end{enumerate}


\section{Proof of the necessary and sufficient condition of cumulative function}\label{sec:proofcfnscondition}

\end{document}




