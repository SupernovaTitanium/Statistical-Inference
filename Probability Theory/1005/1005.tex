\documentclass[../Probability_Theory.tex]{subfiles}
\begin{document}
	\begin{center}
		\renewcommand{\arraystretch}{2}
		\begin{bfseries}
			\begin{tabular}{|c|}
				\hline
				Statistical Inference I \hfill Prof. Chin-Tsang Chiang\\
				\hspace{15em} {\large Lecture Notes 6} \hspace{15em}\ \\
				\lecdate \hfill Scribe: \scribe\\
				\hline
			\end{tabular}
			\renewcommand{\arraystretch}{1}
		\end{bfseries}
	\end{center}


\section{Density function}
\begin{theorem}
$P(X=x)=F(x)-F(x^-)$, where $F(x^-)=\lim_{y\uparrow x}F(y).$
\end{theorem}
\begin{proof}
$P(X=x)=F(X\leq x)-F(X<x)$
\end{proof}
\begin{theorem}
If F(x) is differentiable, there exists a unique probability density function f(x) such that $F(c)=\int_{-\infty}^c f(x)dx\ \forall c\in \mathcal{R}.$  
\end{theorem}
\begin{proof}
The result is quite trivial, just apply the Fundamental Theorem of Calculus, we get $$F'(x)=f(x)$$ 
\end{proof}
The question arises when considering the physical meaning of $f(x)$, is $f(x)$ a probability measure? Go back to the definition of differentiation, we have
$$F'(x)=\lim_{\bigtriangleup\rightarrow 0}\frac{F(x+\frac{\bigtriangleup}{2})-F(x-\frac{\bigtriangleup}{2})}{\bigtriangleup}=\frac{P((x-\frac{\bigtriangleup}{2},x+\frac{\bigtriangleup}{2}))}{\bigtriangleup}=\frac{Probability}{Interval}$$ 
$f(x)$ is not usually probability measure, it is of intensity/density sense.
\end{document}




