\documentclass[../Probability_Theory.tex]{subfiles}
\begin{document}
	\begin{center}
		\renewcommand{\arraystretch}{2}
		\begin{bfseries}
			\begin{tabular}{|c|}
				\hline
				Statistical Inference I \hfill Prof. Chin-Tsang Chiang\\
				\hspace{15em} {\large Lecture Notes 6} \hspace{15em}\ \\
				\lecdate \hfill Scribe: \scribe\\
				\hline
			\end{tabular}
			\renewcommand{\arraystretch}{1}
		\end{bfseries}
	\end{center}

T.B.D
\section{Density function}
\begin{theorem}
$P(X=x)=F(x)-F(x^-)$, where $F(x^-)=\lim_{y\uparrow x}F(y).$
\end{theorem}
\begin{proof}
Since $P(X=x)=F(X\leq x)-F(X<x)$ and notice that  $y\downarrow x$ then $\{X\leq y\}\downarrow\{X\leq x\}$ and $y\uparrow x$ then $\{X\leq y\}\uparrow\{X<x \}$,we have $$P(X=x)=F(x)-F(x^-)$$
\end{proof}
When X is a discrete random variable, $F(x)$ is a step function which jumps at possible outcome.
\begin{theorem}
If F(x) is differentiable, there exists a unique probability density function f(x) such that $F(c)=\int_{-\infty}^c f(x)dx\ \forall c\in \mathcal{R}.$  
\end{theorem}
\begin{proof}
The result is quite trivial, just apply the Fundamental Theorem of Calculus, we get $$F'(x)=f(x)$$ 
\end{proof}
The question arises when considering the physical meaning of $f(x)$, is $f(x)$ a probability measure? Go back to the definition of differentiation, we have
$$F'(x)=\lim_{\bigtriangleup\rightarrow 0}\frac{F(x+\frac{\bigtriangleup}{2})-F(x-\frac{\bigtriangleup}{2})}{\bigtriangleup}=\frac{P((x-\frac{\bigtriangleup}{2},x+\frac{\bigtriangleup}{2}))}{\bigtriangleup}=\frac{Probability}{Interval}$$ 
$f(x)$ is not usually probability measure, it is of intensity/density sense.

\section{Quantative Description of Poisson Random Variable}
Q: Please {\bf quantitatively} describe the Poisson random variable.
\begin{itemize}
	\item It's a {\bf counting process}. That is, $N(t)$ that counts the number of appearances before time $t$.
	\item ({\bf Boundary condition}) $N(0)=0$
	\item ({\bf Stationary}) $\forall t_1<t_2$, $N(t_2)-N(t_1)\sim N(t_2-t_1)$
	\item ({\bf Independence}) $\forall t_1<t_2<t_3<t_4$, $N(t_4)-N(t_3)\sim N(t_2)-N(t_1)$
	\item ({\bf Fixed frequency}) $\lim_{\Delta\rightarrow0^+} \frac{Pr[N(\Delta)-N(0) = 1]}{\Delta}=\lambda$, and $\lim_{\Delta\rightarrow0^+} \frac{Pr[N(\Delta)-N(0) > 1]}{\Delta}=0$
	\item ({\bf Density function}) $f_{\lambda}(t,k) = \frac{(\lambda t)^{k}e^{-\lambda t}}{k!}\mathbf{1}_{\{k=0,1,2,...\}}$
\end{itemize}

\end{document}




